\documentclass{article}
\usepackage{amsmath}

\newcommand{\hatvec}[1]{\hat{\vec{#1}}}
\renewcommand{\vec}[1]{\mathbf{#1}}

\begin{document}

A ray beginning at $\vec r_0$ travelling in the direction of $\vec v$.
A sphere at $\vec s$, with radius $R$.

\begin{align*}
  \vec r(t) &= \vec r_0 + \hatvec v t \\
  \textrm{dist}^2
  &= (\vec s - \vec r(t)) \cdot (\vec s - \vec r(t)) \\
  &= s^2 - 2\vec s \cdot \vec r(t) + r(t)^2 \\
  &= s^2 - 2\vec s \cdot (\vec r_0 + \hatvec v t) + (\vec r_0 + \hatvec v t)^2 \\
  &= s^2 - 2\vec s \cdot \vec r_0 - 2\vec s \cdot \hatvec v t
        + r_0^2 + 2\vec r_0 \cdot \hatvec v t + t^2 \\
  &= s^2 - 2\vec s \cdot \vec r_0 + r_0^2
        + 2(\vec r_0-\vec s) \cdot \hatvec v t + t^2 \\
\end{align*}

Finding the time of closest approach.

\begin{align*}
  \frac{d\textrm{dist}^2}{dt}
  &= 2(\vec r_0 - \vec s) \cdot \hatvec v + 2t = 0 \\
  t_{min} &= (\vec s - \vec r_0) \cdot \hatvec v
\end{align*}

Finding the location of closest approach.
\begin{align*}
  r(t_{min})
  &= \vec r_0 + \hatvec v t_{min} \\
  &= \vec r_0 + \hatvec v \left(\hatvec v \cdot (\vec s - \vec r_0) \right) \\
\end{align*}

Finding the square of the distance of closest approach.
\begin{align*}
  \textrm{dist}^2
  &= \left(\vec r_0 - \vec s
            + \hatvec v \left(\hatvec v \cdot (\vec s - \vec r_0) \right)\right)^2 \\
  &= (\vec r_0 - \vec s)^2
       - 2\left((\vec r_0 - \vec s)\cdot \hatvec v\right)^2
            + \left(\hatvec v \cdot (\vec s - \vec r_0) \right)^2 \\
  &= (\vec r_0 - \vec s)^2
       - \left((\vec r_0 - \vec s)\cdot \hatvec v\right)^2
\end{align*}

When reflecting, first find the component perpendicular to the surface.
Then add twice that amount to the vector to reflect.
\begin{align*}
  \vec v_f &= \vec v_0 + 2(\vec v \cdot (\vec r_{min}-\vec s))(\vec r_{min}-\vec s)
\end{align*}

Remake of the intersection logic.
The previous iteration assumed that the ray would reach the distance of minimum approach.
Instead, I should look for when the approach is exactly the radius.
I assume that $\vec v$ is a unit vector.
\begin{align*}
  \vec r(t) &= \vec r_0 + \vec v t \\
  \textrm{dist} &= \left| \vec r(t) - \vec s \right| = R \\
  R^2
  &= (\vec r(t) - \vec s)^2 \\
  &= (\vec v t + \vec r_0 - \vec s)^2 \\
  &= (\vec v \cdot \vec v)t^2 + 2\vec v \cdot (\vec r_0-\vec s)t + (\vec r_0 - \vec s)^2 \\
  0 &= t^2 + 2\vec v \cdot (\vec r_0-\vec s)t + (\vec r_0 - \vec s)^2 - R^2 \\
  &\left\{\begin{aligned}
      B &= 2\vec v \cdot (\vec r_0-\vec s) \\
      C &= (\vec r_0 - \vec s)^2 - R^2 \\
      t &= \frac{-B \pm \sqrt{B^2-4C}}{2}
  \end{aligned}\right. \\
  &\left\{\begin{aligned}
      B &= \vec v \cdot (\vec r_0-\vec s) \\
      C &= (\vec r_0 - \vec s)^2 - R^2 \\
      t &= -B \pm \sqrt{B^2-C}
  \end{aligned}\right.
\end{align*}

Selection of which root to use:
If $\vec r_0$ is outside the sphere, which should always be the case,
  $C$ is strictly greater than zero.
Therefore, $\sqrt{B^2-C} > |B|$.
Conditions:
If $B>0$, then the ray is moving away from the sphere.
If $C>B^2$, then the ray never intersects with the sphere.
If $B<0$ and $C>B^2$, then $t = -B-\sqrt{B^2-C}$.



\end{document}